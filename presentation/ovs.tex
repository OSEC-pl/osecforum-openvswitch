%
% Copyright (c) 2017 Radoslaw Kujawa.
% All rights reserved.
%
% Redistribution and use in source and binary forms, with or without
% modification, are permitted provided that the following conditions
% are met:
%
% 1. Redistributions of source code must retain the above copyright
%    notice, this list of conditions and the following disclaimer.
% 2. Redistributions in binary form must reproduce the above copyright
%    notice, this list of conditions and the following disclaimer in the
%    documentation and/or other materials provided with the distribution.
%
% THIS SOFTWARE IS PROVIDED BY RADOSLAW KUJAWA (THE AUTHOR) AND CONTRIBUTORS
% ``AS IS'' AND ANY EXPRESS OR IMPLIED WARRANTIES, INCLUDING, BUT NOT LIMITED
% TO, THE IMPLIED WARRANTIES OF MERCHANTABILITY AND FITNESS FOR A PARTICULAR
% PURPOSE ARE DISCLAIMED.  IN NO EVENT SHALL THE AUTHOR OR CONTRIBUTORS
% BE LIABLE FOR ANY DIRECT, INDIRECT, INCIDENTAL, SPECIAL, EXEMPLARY, OR
% CONSEQUENTIAL DAMAGES (INCLUDING, BUT NOT LIMITED TO, PROCUREMENT OF
% SUBSTITUTE GOODS OR SERVICES; LOSS OF USE, DATA, OR PROFITS; OR BUSINESS
% INTERRUPTION) HOWEVER CAUSED AND ON ANY THEORY OF LIABILITY, WHETHER IN
% CONTRACT, STRICT LIABILITY, OR TORT (INCLUDING NEGLIGENCE OR OTHERWISE)
% ARISING IN ANY WAY OUT OF THE USE OF THIS SOFTWARE, EVEN IF ADVISED OF THE
% POSSIBILITY OF SUCH DAMAGE.
%
% 
\documentclass[dvipsnames,table]{beamer}
\usepackage{polski}

\usetheme{Rochester}
\usecolortheme{orchid}

\usepackage{listings}
\usepackage{ucs}
\usepackage[utf8x]{inputenc}
\usepackage{wasysym}
\usepackage[normalem]{ulem}
\usepackage{amsmath}
\usepackage{hyperref}
\usepackage{tikzsymbols}

\setbeamertemplate{navigation symbols}{}
\setbeamertemplate{caption}[numbered]
\setbeamerfont{caption}{size=\scriptsize}
\setbeamercolor{framenote}{bg=OSEC-red!25}
\setbeamercolor{rednote}{bg=Red!25}
\setbeamercolor{palette primary}{use=structure,fg=white,bg=OSEC-red}
\setbeamercolor{palette secondary}{use=structure,fg=white,bg=OSEC-red2}

\setbeamertemplate{itemize item}{\scriptsize\raise1pt\hbox{\donotcoloroutermaths$\blacktriangleright$}}
\setbeamertemplate{itemize subitem}{\tiny\raise1pt\hbox{\donotcoloroutermaths$\bullet$}}
\setbeamertemplate{itemize subsubitem}{\tiny\raise1pt\hbox{\donotcoloroutermaths{--}}}

\setbeamertemplate{enumerate item}{\insertenumlabel.}
\setbeamertemplate{enumerate subitem}{\insertenumlabel.\insertsubenumlabel}
\setbeamertemplate{enumerate subsubitem}{\insertenumlabel.\insertsubenumlabel.\insertsubsubenumlabel}
\setbeamertemplate{enumerate mini template}{\insertenumlabel}

\setbeamercolor{itemize item}{fg=OSEC-red, bg=OSEC-red}
\setbeamercolor{itemize subitem}{fg=OSEC-red, bg=OSEC-red}
\setbeamercolor{itemize subsubitem}{fg=OSEC-red, bg=OSEC-red}

\setbeamercolor{section number projected}{fg=white,bg=OSEC-red}
\setbeamercolor{subsection number projected}{fg=white,bg=OSEC-red}
\setbeamercolor{button}{bg=OSEC-red,fg=white}

\setbeamertemplate{section in toc}[circle]
\setbeamertemplate{subsection in toc}[square]

\definecolor{OSEC-red}{RGB}{160,29,44}
\definecolor{OSEC-red2}{RGB}{177,76,12}
\hypersetup{colorlinks=true,linkcolor=white,urlcolor=OSEC-red}

\setlength{\tabcolsep}{8pt}
\renewcommand{\arraystretch}{1.2}

\newcommand{\tri}{$\triangleright$ }

\title{Open vSwitch -- lab}
\author{Radosław Kujawa -- radoslaw.kujawa@osec.pl}
\institute{OSEC}

\begin{document}

\begin{frame}
	\titlepage
\end{frame}

\begin{frame}
\frametitle{Open vSwitch}
\begin{itemize}
	\item Elastyczny, zarządzalny wirtualny switch do zastosowań Software Defined Network.
	\item Do zarządzania wykorzystywany jest protokół OpenFlow.
	\item Używany domyślnie do wirtualizacji warstwy 2 sieci w Red Hat OpenStack.
	\item Do separacji warstwy drugiej można wykorzystać szereg mechanizmów (VLAN/VXLAN/GRE...).
\end{itemize}
\end{frame}

\begin{frame}
\frametitle{Przygotowania do instalacji}
\begin{itemize}
	\item Open vSwitch dostępny w większości dystrybucji Linuxa.
	\item W Fedorze paczka {\tt openvswitch}.
	\item Wymagania dotyczące konfiguracji.
	\begin{itemize}
		\item NetworkManager konfliktuje z Open vSwitch.
		\item {\tt systemctl stop NetworkManager}
		\item {\tt systemctl disable NetworkManager}
		\item W naszym obrazie VM jest już wyłączony.
	\end{itemize}
\end{itemize}
\end{frame}

\begin{frame}
\frametitle{Nasze środowisko demonstracyjne}
%\begin{itemize}
%	\item insert rysunek here 
% eth0 będzie naszym interfejsem do zarządzania maszynami, skonfigurowany poza openvswitch
%\end{itemize}
\includegraphics[width=1.00\textwidth]{img-labenv.pdf}
\end{frame}

\begin{frame}
\frametitle{Środowisko demonstracyjne -- import maszyn}
\begin{itemize}
	\item Obraz maszyn wirtualnych w formacie OVF 2.0.
	\item Reinitialize the MAC address of all network cards.
	\item Fedora 26 Alpha + Open vSwitch 2.7.0.
	\item Logowanie: {\tt root/fedora}
	\item Nadanie nazwy hosta:
	\item {\tt hostnamectl set-hostname ovshostX.osecforum.pl}
	\item Materiały do labu:
	\item {\tt cd \&\& git clone \href{https://github.com/OSEC-pl/osecforum-openvswitch.git}{https://github.com/OSEC-pl/osecforum-openvswitch.git} lab}
\end{itemize}
\end{frame}

\begin{frame}
\frametitle{Środowisko demonstracyjne -- firewall}
\begin{itemize}
	\item W rzeczywistych warunkach konieczna jest także konfiguracja firewalla (ale tu nie będziemy tego robić).
	\item {\tt systemctl stop firewalld}
	\item {\tt systemctl disable firewalld}
	\item {\tt systemctl mask firewalld}
\end{itemize}
\end{frame}

\begin{frame}
\frametitle{Środowisko demonstracyjne -- konfiguracja}
\begin{itemize}
	\item Przkierowanie połączenia SSH do maszyn wirtualnych:
	\begin{itemize}
		\item ovshostX: Network settings \tri Adapter 1 \tri Port Forwarding.
		\item Host IP: 127.0.0.1
		\item Host port: 1000X
		\item Guest IP: 10.0.2.15
		\item Guest port: 22
		\item Można zalogować się po SSH z systemu operacyjnego hypervisora:
		\item {\tt ssh -p 1000X root@127.0.0.1}
	\end{itemize}
	\item Zezwolenie na przechywywanie ruchu w sieci wewnętrznej:
	\begin{itemize}
		\item ovshostX: Network settings \tri Adapter 2 \tri Promiscuous Mode.
		\item Allow all. 
	\end{itemize}
\end{itemize}
\end{frame}

\begin{frame}
\frametitle{SELinux}
\begin{itemize}
	\item Fedora 26 Alpha wymaga małych poprawek do polityki bezpieczeństwa SELinuxa.
	\item {\tt semodule -i \$HOME/lab/selinux/ovs-f26-fix.pp}
\end{itemize}	
\end{frame}

\begin{frame}
\frametitle{Usługa Open vSwitch}
\begin{itemize}
	\item Paczka: {\tt openvswitch}.
	\item Jednostka systemd: {\tt openvswitch.service} (podnosi też {\tt ovs-vswitchd.service} i {\tt ovsdb-server.service}).
	\item Moduł kernela: {\tt openvswitch}.
	\item {\tt systemctl start openvswitch}
	\item {\tt systemctl enable openvswitch}

\end{itemize}
\end{frame}

\begin{frame}[fragile]
\frametitle{Podstawowa konfiguracja bridge'a}
\begin{itemize}
% disable dhcp for eth1
	\item {\tt ovs-vsctl add-br ovsbr0}
	\item {\tt ovs-vsctl add-port ovsbr0 eth1}
	\item {\tt ovs-vsctl show}
	\item {\tt ip addr add 172.24.254.X/24 dev ovsbr0}
	\item {\tt ip link set dev ovsbr0 up}
\end{itemize}
\scriptsize
\begin{verbatim}
Bridge "ovsbr0"
  Port "eth1"
    Interface "eth1"
  Port "ovsbr0"
    Interface "ovsbr0"
    type: internal
ovs_version: "2.7.0"
\end{verbatim}
\end{frame}

\begin{frame}
\frametitle{Użycie VLAN do separacji sieci warstwy 2}
\begin{itemize}
	\item Wszystkie porty w Open vSwitch domyślnie są trunkami VLAN.
	\item Sytuacja z poprzedniego slajdu -- ruch ,,nieotagowany'' (bez VLAN).
	\item Dodanie interfejsu w konkretnym VLAN:
	\begin{itemize}
		\item {\tt ovs-vsctl add-port ovsbr0 vlan1 -- set interface vlan1 type=internal}
		\item {\tt ovs-vsctl set port vlan1 tag=1}
		\item {\tt ip addr add 172.24.253.X/24 dev vlan1}
		\item {\tt ip link set dev vlan1 up}
	\end{itemize}
\end{itemize}
\end{frame}

\begin{frame}[fragile]
\frametitle{Użycie VLAN do separacji sieci warstwy 2 -- c.d.}
\begin{itemize}
	\item Port {\tt vlan1} jest widoczny na poziomie hosta jako osobny interfejs. 
	\item Ruch wysłany/odbierany na tym interfejsie jest automatycznie tagowany jako VLAN z ID 1.
\end{itemize}
\scriptsize
\begin{verbatim}
Bridge "ovsbr0"
  Port "vlan1"
    tag: 1 
    Interface "vlan1"
      type: internal
  Port "eth1"
    Interface "eth1"
  Port "ovsbr0"
    Interface "ovsbr0"
      type: internal
ovs_version: "2.7.0"
\end{verbatim}	
\end{frame}

\begin{frame}
\frametitle{Tunelowanie ruchu w sieci fizycznej}
\begin{itemize}
	\item Użycie VLAN do separacji na poziomie sieci fizycznej nie zawsze jest możliwe.
	\item Open vSwitch pozwala tunelować ruch między bridge'ami na odległych maszynach.
	\item Konfigurujemy adresację IP eth1 na poziomie systemu operacyjnego.
	\item {\tt ovs-vsctl del-port ovsbr0 eth1}
	\item {\tt ip addr add 192.168.1.X/24 dev eth1}
	\item Stworzenie tunelu GRE:
	\item {\tt ovs-vsctl add-port ovsbr0 ovsgre0 -- set interface ovsgre0 type=gre options:remote\_ip=192.168.1.Y}
\end{itemize}
\end{frame}

\begin{frame}
\frametitle{Tunelowanie ruchu z użyciem VXLAN}
\begin{itemize}
	\item Protokół GRE, jako odrębny protokół IP nie zawsze jest preferowanym rozwiązaniem.
	\item Protokół VXLAN jest bardziej efektywnym rozwiązaniem gdy mamy wiele hostów -- tunelowanie ramek L2 w UDP.
	\item VXLAN wspiera też multicasting... choć sam Open vSwitch jeszcze nie. Implementacja w kernelu Linuxa ma już obsługę multicastingu.
	\item Konfiguracja tunelu VXLAN w Open vSwitch:
	\item {\tt ovs-vsctl add-port ovsbr0 ovsvx0 -- set interface ovsvx0 type=vxlan options:remote\_ip=192.168.1.Y}
\end{itemize}
\end{frame}

\begin{frame}
\frametitle{Network namespaces}
\begin{itemize}
	\item Pozwala na uruchomienie więcej niż jednej ,,instancji'' stosu sieciowego w kernelu Linuxa. 
	\item Każdy namespace ma własne, odrębne interfejsy, własną konfigurację adresacji i tablicę routingu.
	\item Pozwalają na zwirtualizowanie funkcji sieciowych wymagających dostępu do warstwy 2.
	\item Dodanie namespace:
	\item {\tt ip netns add foo}
	\item {\tt ip netns list}
	\item Wykonanie polecenia wewnątrzn namespace:
	\item {\tt ip netns exec foo polecenie}
	\item {\tt ip netns exec foo ip link set dev lo up}
\end{itemize}
\end{frame}

\begin{frame}
\frametitle{Interfejsy {\tt veth}}
\begin{itemize}
	\item Wirtualny Ethernet -- posiada dwie ,,końcówki''. 
	\item Może posłużyć do połączenia namespace z Open vSwitch.
	\item Utworzenie interfejsów {\tt veth}:
	\item {\tt ip link add veth0 type veth peer name veth1}
	\item {\tt ip link set veth1 netns foo}
	\item {\tt ip netns exec foo ip link set dev veth1 up }
	\item {\tt ip netns exec foo ip addr add 172.24.252.X/24 dev veth1} 
\end{itemize}
\end{frame}

\begin{frame}[fragile]
\frametitle{Dodanie {\tt veth} do Open vSwitch jako portu}
\begin{itemize}
	\item {\tt ip link set dev veth0 up}
	\item {\tt ovs-vsctl add-port ovsbr0 veth0}
\end{itemize}
\scriptsize
\begin{verbatim}
Bridge "ovsbr0"
  Port "ovsvx0"
    Interface "ovsvx0"
      type: vxlan
      options: {remote_ip="192.168.1.2"}
    Port "veth0"
      tag: 2
      Interface "veth0"
\end{verbatim}
\normalsize
\begin{itemize}
	\item {\tt ip netns exec foo ping 172.24.252.Y}
	\item {\tt ip netns exec foo ping6 fe80::...\%veth1}
\end{itemize}
\end{frame}

\begin{frame}
\frametitle{Stała konfiguracja}
\begin{itemize}
	\item Konfiguracja bridge'y i wewnętrznych portów z założenia jest stała, konfiguracja adresacji IP {\em nie jest}.
	\item Gdy nie mamy kontrolera SDN, możemy skonfigurować Open vSwitcha przy starcie za pomocą skryptów w katalogu {\tt /etc/sysconfig/network-scripts/}.
	\item OpenStack składa w ten sposób tylko bridge {\tt br-ex}, reszta jest konfigurowana agentem Neutrona. 
	\item Nie wszystkie możliwe rodzaje konfiguracji są wspierane.
	\item Należy pamiętać o odpowiedniej konfiguracji firewalla.
\end{itemize}
\end{frame}

\begin{frame}[fragile]
\frametitle{Stała konfiguracja c.d.}
\begin{itemize}
	\item Przykład: fizyczny interfejs do Open vSwitcha.
\end{itemize}
\begin{columns}
	\begin{column}{0.5\textwidth}
		\begin{verbatim}
		# ifcfg-ovsbr0
		DEVICE=ovsbr0
		ONBOOT=yes
		DEVICETYPE=ovs
		TYPE=OVSBridge
		BOOTPROTO=static
		IPADDR=X.X.X.X
		PREFIX=XX
		\end{verbatim}
	\end{column}
	\begin{column}{0.5\textwidth}
		\begin{verbatim}
		# ifcfg-eth1
		DEVICE=eth1
		ONBOOT=yes
		DEVICETYPE=ovs
		TYPE=OVSPort
		BOOTPROTO=none
		OVS_BRIDGE=ovsbr0
		\end{verbatim}
	\end{column}
\end{columns}
\end{frame}

\begin{frame}[fragile]
\frametitle{Stała konfiguracja c.d.}
\begin{itemize}
	\item Przykład: enkapsulacja sieci L2 w protokół GRE.
\end{itemize}
\begin{columns}
	\begin{column}{0.5\textwidth}
		\begin{verbatim}
	# ifcfg-ovsbr0
	DEVICE=ovsbr0
	ONBOOT=yes
	DEVICETYPE=ovs
	TYPE=OVSBridge
	BOOTPROTO=static
	IPADDR=X.X.X.X
	PREFIX=XX
		\end{verbatim}
	\end{column}
	\begin{column}{0.5\textwidth}
		\begin{verbatim}
	# ifcfg-ovsgre0
	DEVICE=ovsgre0
	ONBOOT=yes
	DEVICETYPE=ovs
	TYPE=OVSTunnel
	OVS_BRIDGE=ovsbr0
	OVS_TUNNEL_TYPE=gre
	OVS_TUNNEL_OPTIONS=
	"options:remote_ip=Y.Y.Y.Y"
		\end{verbatim}
	\end{column}
\end{columns}
\begin{itemize}
	\item Ponadto standardowe interfejsy ethernetowe w sieci, po której przesyłany jest tunel.
\end{itemize}
\end{frame}

\begin{frame}
\frametitle{Stała konfiguracja c.d.}
\begin{itemize}
	\item Wirtualizacja -- interfejsy tap maszyn wirtualnych ({\tt vnetX} w {\tt libvirtd}). Tak samo jak każdy inny interfejs tam w Open vSwitch.
	\item Network namespace -- brak standardu dla statycznej konfiguracji. 
	\item Co z {\tt veth}? \href{https://github.com/LanetNetwork/initscripts-veth}{https://github.com/LanetNetwork/initscripts-veth} (dla Linux Bridge, potrzebne poprawki dla Open vSwitch).
	\item Pole dla własnej inwencji twórczej...
	\item W środowisku korporacyjnym lepiej użyć kontrolera SDN -- niech on konfiguruje Open vSwitcha.
\end{itemize}
\end{frame}

\begin{frame}
\frametitle{Podsumowując...}
\begin{itemize}
	\item Open vSwitch w połączeniu z namespace oraz interfejsami {\tt veth} daje nam niespotykane wcześniej możliwości wirtualizacji sieci oraz funkcji sieciowych.
	\item Pozwala na łatwe wdrożenie aplikacji wymagających bezpośrednio dostępu do warstwy drugiej sieci.
	\item Rozwiązanie to chętnie jest wykorzystywane przy tworzeniu chmur prywatnych (np. wirtualne routery, usługi sieciowe w OpenStack).
\end{itemize}
\end{frame}

\begin{frame}
\frametitle{Koniec\ldots}
\begin{center}
\includegraphics[scale=0.5]{img-oseclogo.png}

Dziękuje!

Czy są pytania?

\end{center}
\end{frame}
\end{document}

