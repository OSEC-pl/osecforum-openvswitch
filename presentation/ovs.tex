%
% Copyright (c) 2017 Radoslaw Kujawa.
% All rights reserved.
%
% Redistribution and use in source and binary forms, with or without
% modification, are permitted provided that the following conditions
% are met:
%
% 1. Redistributions of source code must retain the above copyright
%    notice, this list of conditions and the following disclaimer.
% 2. Redistributions in binary form must reproduce the above copyright
%    notice, this list of conditions and the following disclaimer in the
%    documentation and/or other materials provided with the distribution.
%
% THIS SOFTWARE IS PROVIDED BY RADOSLAW KUJAWA (THE AUTHOR) AND CONTRIBUTORS
% ``AS IS'' AND ANY EXPRESS OR IMPLIED WARRANTIES, INCLUDING, BUT NOT LIMITED
% TO, THE IMPLIED WARRANTIES OF MERCHANTABILITY AND FITNESS FOR A PARTICULAR
% PURPOSE ARE DISCLAIMED.  IN NO EVENT SHALL THE AUTHOR OR CONTRIBUTORS
% BE LIABLE FOR ANY DIRECT, INDIRECT, INCIDENTAL, SPECIAL, EXEMPLARY, OR
% CONSEQUENTIAL DAMAGES (INCLUDING, BUT NOT LIMITED TO, PROCUREMENT OF
% SUBSTITUTE GOODS OR SERVICES; LOSS OF USE, DATA, OR PROFITS; OR BUSINESS
% INTERRUPTION) HOWEVER CAUSED AND ON ANY THEORY OF LIABILITY, WHETHER IN
% CONTRACT, STRICT LIABILITY, OR TORT (INCLUDING NEGLIGENCE OR OTHERWISE)
% ARISING IN ANY WAY OUT OF THE USE OF THIS SOFTWARE, EVEN IF ADVISED OF THE
% POSSIBILITY OF SUCH DAMAGE.
%
% 
\documentclass[dvipsnames,table]{beamer}
\usepackage{polski}

\usetheme{Rochester}
\usecolortheme{orchid}

\usepackage{listings}
\usepackage{ucs}
\usepackage[utf8x]{inputenc}
\usepackage{wasysym}
\usepackage[normalem]{ulem}
\usepackage{amsmath}
\usepackage{hyperref}
\usepackage{tikzsymbols}

\setbeamertemplate{navigation symbols}{}
\setbeamertemplate{caption}[numbered]
\setbeamerfont{caption}{size=\scriptsize}
\setbeamercolor{framenote}{bg=OSEC-red!25}
\setbeamercolor{rednote}{bg=Red!25}
\setbeamercolor{palette primary}{use=structure,fg=white,bg=OSEC-red}
\setbeamercolor{palette secondary}{use=structure,fg=white,bg=OSEC-red2}

\setbeamertemplate{itemize item}{\scriptsize\raise1pt\hbox{\donotcoloroutermaths$\blacktriangleright$}}
\setbeamertemplate{itemize subitem}{\tiny\raise1pt\hbox{\donotcoloroutermaths$\bullet$}}
\setbeamertemplate{itemize subsubitem}{\tiny\raise1pt\hbox{\donotcoloroutermaths{--}}}

\setbeamertemplate{enumerate item}{\insertenumlabel.}
\setbeamertemplate{enumerate subitem}{\insertenumlabel.\insertsubenumlabel}
\setbeamertemplate{enumerate subsubitem}{\insertenumlabel.\insertsubenumlabel.\insertsubsubenumlabel}
\setbeamertemplate{enumerate mini template}{\insertenumlabel}

\setbeamercolor{itemize item}{fg=OSEC-red, bg=OSEC-red}
\setbeamercolor{itemize subitem}{fg=OSEC-red, bg=OSEC-red}
\setbeamercolor{itemize subsubitem}{fg=OSEC-red, bg=OSEC-red}

\setbeamercolor{section number projected}{fg=white,bg=OSEC-red}
\setbeamercolor{subsection number projected}{fg=white,bg=OSEC-red}
\setbeamercolor{button}{bg=OSEC-red,fg=white}

\setbeamertemplate{section in toc}[circle]
\setbeamertemplate{subsection in toc}[square]

\definecolor{OSEC-red}{RGB}{160,29,44}
\definecolor{OSEC-red2}{RGB}{177,76,12}
\hypersetup{colorlinks=true,linkcolor=white,urlcolor=OSEC-red}

\setlength{\tabcolsep}{8pt}
\renewcommand{\arraystretch}{1.2}

\title{Open vSwitch -- lab}
\author{Radosław Kujawa -- radoslaw.kujawa@osec.pl}
\institute{OSEC}

\begin{document}

\begin{frame}
	\titlepage
\end{frame}

\begin{frame}
\frametitle{Open vSwitch}
\begin{itemize}
	\item Wirtualny switch / bridge.
	\item OpenFlow.
	\item Używany domyślnie do zarządzania warstwą 2 w Red Hat OpenStack.
	\item VLAN/VXLAN/GRE.
\end{itemize}
\end{frame}

\begin{frame}
\frametitle{Przygotowania do instalacji}
\begin{itemize}
	\item Open vSwitch dostępny w większości dystrybucji Linuxa.
	\item W Fedorze paczka {\tt openvswitch}.
	\item Wymagania dotyczące konfiguracji.
	\begin{itemize}
		\item NetworkManager konfliktuje z Open vSwitch.
		\item {\tt systemctl stop NetworkManager}
		\item {\tt systemctl disable NetworkManager}
		\item W naszym obrazie VM jest już wyłączony.
	\end{itemize}
\end{itemize}
\end{frame}

\begin{frame}
\frametitle{Nasze środowisko demonstracyjne}
\begin{itemize}
	\item insert rysunek here 
% eth0 będzie naszym interfejsem do zarządzania maszynami, skonfigurowany poza openvswitch
\end{itemize}
\end{frame}

\begin{frame}
\frametitle{Środowisko demonstracyjne -- import maszyn}
\begin{itemize}
	\item Obraz maszyn wirtualnych w formacie OVF 2.0.
	\item Reinitialize the MAC address of all network cards 
\end{itemize}
\end{frame}

\begin{frame}
\frametitle{Środowisko demonstracyjne -- konfiguracja}
\begin{itemize}
	\item Fedora 26 Alpha.
	\item Open vSwitch 2.7.0.
	\item root/fedora
	\item Materiały do labu:
	\item {\tt cd \&\& git clone \href{https://github.com/OSEC-pl/osecforum-openvswitch.git}{https://github.com/OSEC-pl/osecforum-openvswitch.git} lab}
	\item Nadanie nazwy hosta:
	\item {\tt hostnamectl set-hostname ovshostX.osecforum.pl}
\end{itemize}
\end{frame}

\begin{frame}
\frametitle{Środowisko demonstracyjne -- konfiguracja}
\begin{itemize}
	\item ovshostX: Network settings->Adapter 1->Port Forwarding.
	\item Host IP: 127.0.0.1
	\item Host port: 1000X
	\item Guest IP: 10.0.2.15
	\item Guest port: 22
	\item Można zalogować się po SSH z systemu operacyjnego hypervisora:
	\item {\tt ssh -p 1000X root@127.0.0.1}
\end{itemize}
\end{frame}

\begin{frame}
\frametitle{Usługa Open vSwitch}
\begin{itemize}
	\item Paczka: {\tt openvswitch}.
	\item Jednostka systemd: {\tt openvswitch.service} (podnosi też {\tt ovs-vswitchd.service} i {\tt ovsdb-server.service}).
	\item Moduł kernela: {\tt openvswitch}.
	\item {\tt systemctl start openvswitch}
	\item {\tt systemctl enable openvswitch}

\end{itemize}
\end{frame}

\begin{frame}
\frametitle{SELinux}
\begin{itemize}
	\item Fedora 26 Alpha wymaga małych poprawek do polityki bezpieczeństwa SELinuxa.
	\item {\tt semodule -i \$HOME/lab/selinux/ovs-f26-fix.pp}
\end{itemize}	
\end{frame}

\begin{frame}[fragile]
\frametitle{Podstawowa konfiguracja bridge'a}
\begin{itemize}
% disable dhcp for eth1
	\item {\tt ovs-vsctl add-br ovsbr0}
	\item {\tt ovs-vsctl add-port ovsbr0 eth1}
	\item {\tt ovs-vsctl show}
\end{itemize}
	\scriptsize
\begin{verbatim}
Bridge "ovsbr0"
  Port "eth1"
    Interface "eth1"
  Port "ovsbr0"
    Interface "ovsbr0"
    type: internal
ovs_version: "2.7.0"
\end{verbatim}
\end{frame}
	
\begin{frame}
\frametitle{Stała konfiguracja}
\begin{itemize}
	\item Gdy nie mamy kontrolera SDN, możemy skonfigurować Open vSwitcha przy starcie za pomocą skryptów w katalogu {\tt /etc/sysconfig/network-scripts/}.
	\item OpenStack składa w ten sposób tylko interfejs {\tt br-ex}. 
\end{itemize}
\end{frame}

\begin{frame}
\frametitle{Koniec\ldots}
\begin{center}
\includegraphics[scale=0.5]{img-oseclogo.png}

Dziękuje!

Czy są pytania?

\end{center}
\end{frame}
\end{document}

